\documentclass[a4paper,12pt]{article}

\usepackage{amssymb}
\usepackage{amsmath}
\usepackage{amsfonts}
\usepackage{txfonts}
%\usepackage{upgreek}
\usepackage{graphicx}
%\usepackage{siunitx}
\usepackage{enumerate}
\usepackage[left=2cm,right=2cm,top=2cm,bottom=2cm]{geometry}

%\newcommand{\question}[2]{\textbf{\textit{#1}}\quad{\footnotesize\textit{(#2 points)}}\\[3mm]}
\newcommand{\question}[1]{\textbf{\textit{#1}}}
\newcommand{\points}[1]{\quad{\footnotesize\textit{(#1 points)}}}
\newcommand{\point}{\quad{\footnotesize\textit{(1 point)}}}
\newcommand{\HRule}{\rule{\linewidth}{0.3mm}}
\newcommand{\dd}{\mathrm{d}}
\newcommand{\pip}{\uppi}
\newcommand{\ii}{\mathrm{i}}
\renewcommand{\thefootnote}{\normalsize\fnsymbol{footnote}}
\DeclareMathOperator{\e}{e}
\newcommand{\bra}{\langle}
\newcommand{\ket}{\rangle}

\renewcommand{\theequation}{\Roman{equation}}

\begin{document}
\pagestyle{empty}

\begin{center}
\LARGE \textbf{Astronomy from 4 perspectives: the Dark Universe}
\HRule
\end{center}
\begin{flushright}
prepared by: Heidelberg participants and BMS
\end{flushright}
\begin{center}
{\Large \textbf{Solutions: the planet of the Petit Prince}}
\end{center}
\vspace{5mm}

\begin{enumerate}[\itshape \bfseries 1.]

\item \question{Gravity on the planet of the Petit prince}\\
The Petit Prince by A. de Saint-Exupery lives on a planet which,
according to images, is roughly $R\simeq 1~\mathrm{m}$ in size and
because Saint-Exupery does not provide any other information, has a
value of the surface gravity $g=9.81~\mathrm{m}/\mathrm{s}^2$ similar
to Earth. But in comparison to Earth where the gradient of the
acceleration is almost zero, it is much stronger on the planet of the
Petit Prince. Recall that $G=6.6\times 10^{-11}$ in SI.
\begin{enumerate}[(a)]
\item{The relation between the mass and the density of the planet is
    $M=(4\pi /3)R^3$. The surface gravity is tied to the mass by
    $g=GM/R^2$. Substituting one in the other one can solve to find a
    density $\rho \simeq 3.5\times 10^{10}$ kg m$^{-3}$. This is quite
  close to the density of a White Dwarf.}
\item{The orbital period and velocity can be obtained by equation the
    grvitational accelelration to the centrifugal acceleration:
    $GM/(R+1\mathrm{m})^2 = \Omega^2 (R+1\mathrm{m}) $. This gives
    $\Omega \simeq 1$ s$^{-1}$ corresponding to a period $P\simeq 6$ sec, and
    an orbital speed $V=\Omega (R+1\mathrm{m})\simeq 4$ m
    s$^{-1}$. Yes the Petit Prince can throw an object fast enough to
    put it into orbital motion.}
\item{The escape speed is given by equating the specific kinetic
    energy $V^2/2$ to the potential energy $GM/R$, and this gives a
    typical value $V\simeq 4$ m
    s$^{-1}$. This is too much for a kid to jump.}
\item{Given that the maximum period is 6 sec (computed above, and our
    day corresponds to $86400$ sec then
    there will be at most $14000$ sunsets/sunrises.}
\end{enumerate}

\item \question{Devices on the planet of the Petit prince}\\
Imagine that Saint-Exupery brings simple mechanical systems with him, and find out if they behave differently because of the strong gradient $\partial g/\partial r$ in the gravitational acceleration $g$.
\begin{enumerate}[(a)]
\item{What's the relation between the oscillation period $T$ of a pendulum clock as a function of height $h$? Would the oscillation period be independent from the amplitude?}
\item{Saint-Exupery and the Petit Prince have a glass of orange juice
    with an ice cube. The Petit Prince's ice cube swims higher or not above the surface of the juice compared to Saint-Exupery's?}
\end{enumerate}

\item \question{relativity on the planet of the Petit prince}\\
Are there relativistic effects of gravity on the planet of the Petit Prince?
\begin{enumerate}[(a)]
\item{What is the tidal gravitational acceleration between the head
    and the feet of the Petit Prince? Please compute the difference
\begin{equation}
\Delta g = \frac{GM}{R^2}-\frac{GM}{(R+1)^2} 
\end{equation}
}
\item{What is the gravitational time dilation between the head and the feet of the Petit Prince? Please use the formula
\begin{equation}
\Delta \tau = \sqrt{1+2\frac{\Phi}{c^2}}\:\Delta t
\end{equation}
and approximate the potential as homogeneous, $\Phi = g\Delta r$.
}
\end{enumerate}
\end{enumerate}
\end{document}
