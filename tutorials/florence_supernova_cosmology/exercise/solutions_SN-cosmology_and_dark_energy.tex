\documentclass[a4paper,12pt]{article}

\usepackage{amssymb}
\usepackage{amsmath}
\usepackage{amsfonts}
\usepackage{txfonts}
\usepackage{upgreek}
\usepackage{graphicx}
%\usepackage{siunitx}
\usepackage{enumerate}
\usepackage[left=2cm,right=2cm,top=2cm,bottom=2cm]{geometry}

%\newcommand{\question}[2]{\textbf{\textit{#1}}\quad{\footnotesize\textit{(#2 points)}}\\[3mm]}
\newcommand{\question}[1]{\textbf{\textit{#1}}}
\newcommand{\points}[1]{\quad{\footnotesize\textit{(#1 points)}}}
\newcommand{\point}{\quad{\footnotesize\textit{(1 point)}}}
\newcommand{\HRule}{\rule{\linewidth}{0.3mm}}
\newcommand{\dd}{\mathrm{d}}
\renewcommand{\pi}{\uppi}
\newcommand{\ii}{\mathrm{i}}
\renewcommand{\thefootnote}{\normalsize\fnsymbol{footnote}}
\DeclareMathOperator{\e}{e}

\renewcommand{\theequation}{\Roman{equation}}

\begin{document}
	\pagestyle{empty}
	
	\begin{center}
		\LARGE \textbf{Astronomy from 4 perspectives: the Dark Universe}
		\HRule
	\end{center}
	\begin{flushright}
		prepared by: Florence participants and BMS
	\end{flushright}
	\begin{center}
		{\Large \textbf{exercise: Supernova-cosmology and dark energy}}\\
		\vspace*{2mm}
		{\Large \textbf{Solutions}}
		
	\end{center}
	\vspace{5mm}
	
	\begin{enumerate}[\itshape \bfseries 1.]
		
		\item \question{light-propagation in FLRW-spacetimes}\\
		Photons travel along null geodesics, $\dd s^2=0$, in any spacetime. 
		\begin{enumerate}[(a)]
			\item Let us do the following substitution $dt
                          \rightarrow a(t)d\tau$ then the line element
                          can be written
                          $ds^2=a(t)^2[c^2d\tau^2-d\chi^2]$ and the
                          equation ofr the null-geodesic will be
                          $d\chi=\pm cd\tau$.
			\item The cosmic time is the time measured by
                          a cosmic observer synchronized for $t=0$ 
 \begin{align}
   t=\int_0^t dt' = \int_0^a\frac{da'}{\dot{a}'}
   \end{align}
The conformal time is tied to the time interval over which an observer
at $t=t_0$ sees to happen an event in the past at time $t$. Now at
$t=t_0$ this will coincide with the cosmic time, ence it will be
afected by cosmic time dilation.
                          \begin{align}
                            \tau(t)=\int_0^t \frac{dt'}{a(t')} = \frac{1}{a(t)}
                              \int_0^t \frac{a(t)}{a(t')} dt' > \frac{t}{a(t)}
                          \end{align}
			\item Now for the given metric:
                          \begin{align}
                            H=\frac{\dot{a}}{a}= H_o a^{-3(1+w)/2}
                            \Rightarrow \dot{a} = H_o
                            a^{1-3(1+w)/2}\\
                            a(t)=\left(\frac{4}{9}\right)^{3(w+1)}(t+wt)^{2/3(w+1)}
                            \end{align}
                          \begin{align}
                            \tau_H=\int_0^t \frac{dt'}{a(t')} =
                            \int_0^1 \frac{da'}{\dot{a}a} =
                            \frac{1}{H_o}\int_0^1 a^{3(w+1)/2 -2}da =  \frac{1}{H_o}\frac{1}{3(w+1)/2 -1}
                            \end{align}
			\item Isotropy of the universe ensures us that
                          it is not.
		\end{enumerate}
		
		
		\item \question{light-propagation in perturbed metrics}\\
		\begin{align}
		ds^2=\left(1+2\frac{\Phi}{c^2}\right)c^2dt^2-\left(1-2\frac{\Phi}{c^2}\right)dx^2 
		\end{align}
		With $ds^2=0$:
		\begin{align}
		\left( 1+\frac{2\Phi}{c^2}\right)c^2dt^2 &= \left(1-\frac{2\Phi}{c^2}\right) dx^2\\
		\frac{dx}{dt}&=\pm c\sqrt{\frac{1+\frac{2\Phi}{c^2}}{1-\frac{2\Phi}{c^2}}}
		\end{align}
		With $\frac{1}{1-\epsilon}\approx 1+\epsilon$ for small $\epsilon$:
		\begin{align}
		\frac{dx}{dt}\approx\pm c\left(1+\frac{2\Phi}{c^2}\right)
		\end{align} 
		For a non-zero $\Phi$ this is not equal to $c$! \\
		We assign an effective index of refraction by:
		\begin{align}
		n(\Phi)=\frac{dx/dt}{c}\approx \left(1+\frac{2\Phi}{c^2}\right)
		\end{align}
		
		\item \question{classical potentials including a cosmological constant}\\
		The field equation of classical gravity including a
                cosmological dark energy density $\lambda$ is given by
		\begin{equation}
		\Delta\Phi = 4\pi G(\rho + \lambda)
		\end{equation}
		(a) field calculation\\
	
		Now it is possible to simply integrate the field equation starting with:
		\begin{align}
		  \Delta\Phi&=\frac{1}{r^{2}}\frac{\partial}{\partial r}\left(r^{2}\frac{\partial\Phi}{\partial r}\right)\\
		  &=4\pi G(\rho(r)+\lambda)\\
		  r^{2}\frac{\partial\Phi}{\partial
                    r}&=\int_0^r\textrm{d}r'\left(4\pi G[\left(r'\right)^{2}\rho\left(r'\right)+\left(r'\right)^{2}\lambda]\right)\\
		  &=GM+G\frac{\lambda}{3}r^3\\
		  \frac{\partial\Phi}{\partial r}&=\frac{GM}{r^{n-1}}+G\frac{\lambda r}{n}\\
		  \Phi&=-\frac{GM}{r}+G\frac{\lambda r^2}{6}
		\end{align}
		(b) power-law solutions\\
		Following the calculation one may see that each source term corresponds to an individual power-law:
		\begin{align*}
		  C(n)G\rho(r) ~~~ &\Rightarrow ~~~ -\frac{GM}{r}\\
		  \lambda ~~~ &\Rightarrow ~~~ G\frac{\lambda r^2}{6}
		\end{align*}
		(c) equilibrium\\
		To find an equilibrium distance one must set \(\Phi\left(r_\textrm{eq}\right)=0\)
		\end{enumerate}
		\begin{align}
		  \frac{GM}{r_\textrm{eq}}&=G\frac{\lambda r_\textrm{eq}^2}{6}\\
		  \frac{\lambda r_\textrm{eq}^3}{6}&=M
		\end{align}
		from which follows immediatly:
		\begin{equation}
		  r_\textrm{eq}=\sqrt[3]{6\frac{M}{\lambda}}
		\end{equation}
		(c) if one inputs the number one gets $ r_\textrm{eq}
                =1.5$ Mpc one hundred times larger than the size of a galaxy.
        \begin{enumerate}
        \setcounter{enumi}{3}
		\item \question{physics close to the horizon}\\
		Why is it necessary to observe supernovae at the Hubble distance $c/H_0$ to see the dimming in accelerated cosmologies? Please start at considering the curvature scale of the Universe: A convenient quantisation of curvature might be the Ricci-scalar $R = 6H^2(1-q)$ for flat FLRW-models.
		\begin{enumerate}[(a)]
			\item The Dimension of the Ricci-scalar is $1/s^2$ thus we can define a time and a distance scale by:
			$$ \tau = 1/\sqrt{R} \ \ \ \mathrm{and} \ \ d = c/\sqrt{R} \approx c/H_0$$
			which gives the curvature scale of the Universe. 
			\item To observe supernovae dimming caused by accelerated cosmic expansion the supernova distance had to be about (or larger than) the curvature scale, because at distances $<<d$ the different cosmological distance measures converge. \\
			For illustration see: \textit{https://en.wikipedia.org/wiki/Distance\_measures\_(cosmology)}
		\end{enumerate}
		
		
		\item \question{measure cosmic acceleration}\\
		The luminosity distance $d_\mathrm{lum}(z)$ in a spatially flat FLRW-universe is given by
		\begin{equation}
		d_\mathrm{lum}(z) = (1+z)\int_0^z\mathrm{d}z^\prime\:\frac{1}{H(z^\prime)}
		\end{equation}
		with the Hubble function $H(z)$. Let's assume that the Universe is filled with a cosmological fluid up to the critical density with a fluid with equation of state $w$, such that the Hubble function is
		\begin{equation}
		H(z) = H_0 (1+z)^\frac{3(1+w)}{2}.
		\end{equation}
		\begin{enumerate}
			\item
					By definition:
					\begin{align*}
	H=\frac{\dot{a}}{a} \text{ and } q=-\frac{\ddot{a}a}{\dot{a}^2}
		\end{align*}
		It follows
		\begin{align*}
		\dot{H}&=\frac{\ddot{a}a-\dot{a}^2}{a^2}=\frac{\ddot{a}a}{a^2}-H^2		
		\end{align*}
		So we get
		\begin{align*}
		\frac{\dot{H}}{H^2}&=\frac{\ddot{a}a}{\dot{a}^2}-1=-q-1\\
		q&=-(\frac{\dot{H}}{H^2}+1)	
		\end{align*}
	
		
		We also have
		\begin{align*}
		H=H_0\cdot(1+z)^{\frac{3(1+w)}{2}}=H_0\cdot a^{\frac{-3(1+w)}{2}}
		\end{align*}
		and
		\begin{align*}
		\dot{H}&=H_0\left(\frac{-3(1+w)}{2}\right)\cdot a^{\frac{-3(1+w)}{2}}\cdot \dot{a}\\
			  &=H_0\cdot a^{\frac{-3(1+w)}{2}}\cdot\frac{\dot{a}}{a}\cdot\left(\frac{-3(1+w)}{2}\right)\\
			  &=H^2\cdot \left(\frac{-3(1+w)}{2}\right)
		\end{align*}	
		so
		\begin{align*}
		q=-\left(\frac{-3(1+w)}{2}+1\right)=\frac{1}{2}(3w+1)
		\end{align*}
		and obviously
		\begin{align*}
		q<0 \text{ for } w<-\frac{1}{3}\\
		q>0 \text{ for } w>-\frac{1}{3}
		\end{align*}	
										
			\item
				First, we consider the case $w=-\frac{1}{3}$ (non-accelerating universe):
				\begin{align*}
			H=H_0(1+z)^{\frac{3(1+w)}{2}}=H_0(1+z)\\
\end{align*}	
\begin{align*}
d_{lum,1}&=(1+z)\int_0^z \! \frac{1}{H(z')} \, \mathrm{d}z'\\
			&=(1+z)\int_0^z \! \frac{1}{H_0(1+z')} \, \mathrm{d}z'\\
			&=\frac{1+z}{H_0}ln(1+z)
\end{align*}						
	Now, we consider the case $w<-\frac{1}{3}$ (accelerating universe):
	\begin{align*}
				d_{lum,2}&=(1+z)\int_0^z \! \frac{1}{H(z')} \, \mathrm{d}z'\\
				&=\frac{1+z}{H_0}\int_0^z \! (1+z')^{\frac{-3(1+w)}{2}} \, \mathrm{d}z'\\
				&=\frac{1+z}{H_0}\left[(1+z')^{\frac{-3(1+w)+2}{2}}\cdot \frac{2}{-3(1+w)+2}\right]_0^z\\
				&=\frac{1+z}{H_0}\left(\frac{2}{-3(1+w)+2}\right)\left[(1+z')^{\frac{-3(1+w)+2}{2}}-1\right]
\end{align*}	
It follows:
$d_{lum_2}(z)>d_{lum_1}(z)$, because the exponent $\frac{-3(1+w)+2}{2}$ is positive $(w<-\frac{1}{3})$,
so $d_{lum_2}(z)$ is growing faster, than the logarithmic function $d_{lum_1}(z)$.		
			\item Yes, as long as the universe is flat. Just try to plot the two functions
			\item The formula still apply,. But in a
                          contracting universe one would see light
                          that is blue-shifted and not red-shifted. Su
                          $z<0$. But Obviiously $z>-1$ otherwhise one
                          would get negative wavelengths (frequancies)
                          that make no sense. As a consequence, the
                          origin of time in a contracting universe
                          cosrresponds to $z=-1$ for a universe
                          contracting from infinity. Hence the limit
                          of integration must be corrected accordingly.
		\end{enumerate}
		
	\end{enumerate}
\end{document}
