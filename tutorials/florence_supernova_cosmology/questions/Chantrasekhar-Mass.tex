\documentclass[a4paper,12pt]{article}

\usepackage{amssymb}
\usepackage{amsmath}
\usepackage{amsfonts}
\usepackage{txfonts}
%\usepackage{upgreek}
\usepackage{graphicx}
%\usepackage{siunitx}
\usepackage{enumerate}
\usepackage[left=2cm,right=2cm,top=2cm,bottom=2cm]{geometry}

%\newcommand{\question}[2]{\textbf{\textit{#1}}\quad{\footnotesize\textit{(#2 points)}}\\[3mm]}
\newcommand{\question}[1]{\textbf{\textit{#1}}}
\newcommand{\points}[1]{\quad{\footnotesize\textit{(#1 points)}}}
\newcommand{\point}{\quad{\footnotesize\textit{(1 point)}}}
\newcommand{\HRule}{\rule{\linewidth}{0.3mm}}
\newcommand{\dd}{\mathrm{d}}
\renewcommand{\pi}{\uppi}
\newcommand{\ii}{\mathrm{i}}
\renewcommand{\thefootnote}{\normalsize\fnsymbol{footnote}}
\DeclareMathOperator{\e}{e}

\renewcommand{\theequation}{\Roman{equation}}

\begin{document}
\pagestyle{empty}

\begin{center}
\LARGE \textbf{Astronomy from 4 perspectives: the Dark Universe}
\HRule
\end{center}
\begin{flushright}
prepared by: Bucciantini N.
\end{flushright}
\begin{center}
{\Large \textbf{Quantum Mechanics, Relativity and Supernovae}}
\end{center}
This tutorial is aimed at advanced students, to whom the teacher has
already introduced the basic concepts of quantum mechanics and
relativity. In particular the only notion of QM required is the {\bf
  uncertainty principle of position and momentum} and the only notion
from SR the {\bf relativistic
relation between energy and momentum}. Some familiarity with
thermodynamics and hydrostatic is also required.

We are going to show how with  simples arguments using basic quantum
mechanics and special relativity one can find that there is a limit to
the mass of a quantistic star.\\

The starting points are:
\begin{enumerate}
\item From quantum mechanics the uncertainty principle  of momentum and
position $\Delta x \Delta p
=\hbar$. The teacher should have introduced it before and explained its
meaning and implications.
\item From special relativity the relativistic relation between energy
  and momentum of a relativistic particle $e=cp$. One can use photons
  to show that they carry energy (obvious - think of solar panels that
  produce electricity) and also momentum (the solar mill is a good example).
\item From thermodynamics the relation between pressure of a gas and
  the energy density. This can be introduced classically with a
  discussion of the relation $P = n kT = U$, recalling that $kT$ is
  the thermal energy of a particle of a gas.
\item From hydrostatic the equilibrium relation $F = -\nabla P$,
  stating that any external force (force density to be more precise) must be balanced by a pressure
  gradient. As an example one can discuss how pressure changes going
  under water, by simply showing that at any depth the pressure must
  be equal to the force exerted by the overlying column of water. And
  use this argument to derive the hydrostatic relation. 
\end{enumerate}

We begin by showing how a relativistic gas behaves.\\

First consider a volume V containing N particles. Then the average
volume per particle is $V/N = 1/n$ where we have introduced the
particle density $n$. Then one can define the average distance between particles
is $(V/N)^{1/3} = 1/n^{1/3}$. Using the uncertainty principle, and taking as a
typical uncertainty over the distance the average distance between
particles, we get a typical momentum $p= \hbar n^{1/3}$. Now use the
relativistic energy momentum relation to get a
typical energy $e= c \hbar n^{1/3}$. Finally the energy density of this system of
particles will be $U= n e = c \hbar n^{4/3}$. Recall from classical
thermodynamics that the pressure is of the order of the energy density
then the pressure of our quantistic and relativistic
system is $P= c \hbar n^{4/3}$. And setting equal to $m$ the typical
mass of a particle $P=c \hbar (\rho/m)^{4/3} $, where we have
introduced the mas density $\rho$.

We now turn to hydrostatic equilibrium. For a star the equation at
any depth where the density is $\rho$ will look like:
\begin{align}
G\frac{M\rho}{r^2} = -\nabla P = -\frac{dP}{dr}
\end{align}
given that stars are spherically symmetric. At this point we are going
to simplify the treatments looking only at the scaling of the
equation. This is done replacing some of the quantities with their
simplest approximations: the local radius
$r\rightarrow R$ the stellar radius; the density $\rho \rightarrow M
R^{-3}$ where $M$ is the mass of the star; the pressure derivative
$dP/dr \rightarrow P/R$. Then the differential equation turns into an
algebraic one that the students should be more familiar with.
\begin{align}
G\frac{M^2 R^{-3}}{R^2} = -\frac{P}{R}\\
G\frac{M^2}{R^5}  =\frac{ \hbar c}{R} \left(\frac{M}{mR^3}\right)^{4/3}
\end{align}
The student should see immediately that the radius simplifies out of
the equation. This means that the equilibrium is independent of the
radius. Or stated in other words that if the equilibrium (the above
equation) is not satisfied, the star will start to expand (explode) or
collapse (implode), and you cannot avoid this by adjusting the radius;
it is a catastrophic process. The student should also see that the
equilibrium equation gives the following solution of the mass:
\begin{align}
M_{\rm eq} = \frac{1}{m^2}\left(\frac{c \hbar}{G}   \right)^{3/2}
\end{align}
Now let us put the numbers: the speed of light $c=3\times 10^9$ m
s$^{-1}$; $m$ is the mass of a proton $1.6 \times
10^{-27}$ kg; $G = 6.6 \times 10^{-11}$ m$^3$ s$^{-2}$ kg$^{-1}$;
$\hbar=10^{-34}$ J s. Substituting these values one gets:
\begin{align}
M_{\rm eq} = 3.8 \times 10^{30} {\rm kg}
\end{align}
This is about 2 times that mass of the Sun $M_{\rm Sun} = 2\times
10^{30}$ kg (doing the correct model the astrophysicists get 1.4 times
the mass of the Sun - be happy with such simple equation you got only
a 25\% difference). This equilibrium mass is known as {\bf Chandrasekhar mass}. If
the mass is smaller, the star will expand until the physical processes
of the gas change so much that somehow the star reaches a new
equilibrium. If the stellar mass gets bigger than the gravity wins and the star will
start to collapse, driving a catastrophic evolution, that can either
end with a black hole or with a stellar detonation, known as Supernova.

\vspace{5mm}

\end{document}
