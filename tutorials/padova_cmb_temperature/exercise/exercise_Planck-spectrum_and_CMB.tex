\documentclass[a4paper,12pt]{article}

\usepackage{amssymb}
\usepackage{amsmath}
\usepackage{amsfonts}
\usepackage{txfonts}
\usepackage{upgreek}
\usepackage{graphicx}
\usepackage{siunitx}
\usepackage{enumerate}
\usepackage[left=2cm,right=2cm,top=2cm,bottom=2cm]{geometry}

%\newcommand{\question}[2]{\textbf{\textit{#1}}\quad{\footnotesize\textit{(#2 points)}}\\[3mm]}
\newcommand{\question}[1]{\textbf{\textit{#1}}}
\newcommand{\points}[1]{\quad{\footnotesize\textit{(#1 points)}}}
\newcommand{\point}{\quad{\footnotesize\textit{(1 point)}}}
\newcommand{\HRule}{\rule{\linewidth}{0.3mm}}
\newcommand{\dd}{\mathrm{d}}
\renewcommand{\pi}{\uppi}
\newcommand{\ii}{\mathrm{i}}
\renewcommand{\thefootnote}{\normalsize\fnsymbol{footnote}}
\DeclareMathOperator{\e}{e}
\newcommand{\bra}{\langle}
\newcommand{\ket}{\rangle}

\renewcommand{\theequation}{\Roman{equation}}

\begin{document}
\pagestyle{empty}

\begin{center}
\LARGE \textbf{Astronomy from 4 Perspectives: the Dark Universe}
\HRule
\end{center}
\begin{flushright}
prepared by: Padova participants and Bj{\"o}rn Malte Sch{\"a}fer
\end{flushright}
\begin{center}
{\Large \textbf{Exercise: Planck-spectrum and the CMB}}
\end{center}
\vspace{5mm}

\begin{enumerate}[\itshape \bfseries 1.]

\item \question{Properties of the Planck-spectrum}\\
Let's derive the fundamental properties of the Planck-spectrum,
\begin{equation}
S(\nu) = S_0\frac{\nu^3}{\exp(h\nu/(k_BT))-1}
\quad\rightarrow\quad
S(\nu) = S_0\nu^3\exp(-h\nu/(k_BT)),
\end{equation}
by using Wien's approximation (the second expression), which makes the integrals easier. The constant $S_0$ depends only on numbers, natural and mathematical constants.
\begin{enumerate}[(a)]
\item{Please compute the total intensity $\int_0^\infty\dd\nu\:S(\nu)$ and show that it is $\propto T^4$.}
\item{Show that the position $\nu_m$ of the maximum scales $\nu_m\propto T$.}
\item{Please derive the scaling of the mean
\begin{equation}
\bra\nu\ket = \frac{\int_0^\infty\dd\nu\:\nu S(\nu)}{\int_0^\infty\dd\nu\:S(\nu)}
\end{equation}
and show that it is proportional to $T$.}
\item{Is there a fixed ratio between $\bra\nu\ket$ and $\nu_m$?}
\item{In which limit is Wien's approximation applicable?}
\item{Do the scaling behaviours with $T$ derived above depend on the details of the distribution?}
\end{enumerate}

\item \question{Wien's distribution function}\\
Let's stick for a second with Wien's distribution function in $n$ dimensions,
\begin{equation}
S(\nu) = S_0\nu^n\exp(-h\nu/(k_BT)),
\end{equation}
and derive a few general properties, which will hold for the Planck-distribution as well (although the computations are more complicated).
\begin{enumerate}[(a)]
\item{Please begin by showing that
\begin{equation}
\int_0^\infty\dd x\:x^n\exp(-x) = n!
\end{equation}
using $n$-fold integration by parts.}
\item{Alternatively, please show the recursion relation of the $\Gamma$-function, 
\begin{equation}
\Gamma(n) = (n-1)\Gamma(n-1),
\quad\mathrm{together~with}\quad
\Gamma(0) = 1.
\end{equation}
The $\Gamma$-function is defined by 
\begin{equation}
\Gamma(n) = \int_0^\infty\dd x\:x^{n-1}\exp(-x),
\end{equation}
and is related to the factorial by $\Gamma(n) = (n-1)!$}
\item{What scaling of the moments
\begin{equation}
\bra\nu^m\ket = \frac{\int_0^\infty\dd\nu\:\nu^mS(\nu)}{\int_0^\infty\dd\nu\:S(\nu)}
\end{equation}
with temperature $T$ do you expect?
}
\item{Please show that the skewness parameter $s = \bra\nu^3\ket/\bra\nu^2\ket^{3/2}$ and the kurtosis parameter $k = \bra\nu^4\ket/\bra\nu^2\ket^2$ are independent from the temperature $T$. What would be the physical interpretation of $s$ and $k$?}
\item{Would an equivalent result be true for the parameter $\bra\nu^{2n}\ket/\bra\nu^2\ket^n$?}
\end{enumerate}

\item \question{Planck-spectra at cosmological distances}\\
Imagine you observe an object emitting a Planck-spectrum at a cosmological distance, such that all photons arrive with a redshifted frequency $\nu\rightarrow a\nu=\nu / (1+z)$ with scale factor $a$ (remember $a<1$) and redshift $z$. A couple of students discusses the fact that the temperature scales with $T\propto 1/a$ and that the photons are redshifted: What's your opinion on the different arguments?
\begin{enumerate}[(a)]
\item{Johannes from Heidelberg says: The temperature $T$ of a photon gas is linked to the thermal energy $E$ by $E=k_BT$. Then, the relativistic dispersion relation of the photons assumes $E = cp$ with the momentum $p$. The momentum $p$ is given by the de Broglie-relationship as $p=h/\lambda$. If now the photon wavelength is changed $\lambda\rightarrow a\lambda$, the temperature needs to scale $T\propto 1/a$.}
\item{Antonia from Padova says: What about a purely thermodynamical argument? A gas of photons has an adiabatic index of $\kappa=4/3$, and the Hubble expansion is an adiabatic change of state, because there is no thermal energy created or dissipated. Then, the adiabatic invariant says that $TV^{\kappa-1}$ is conserved, which gives me $T\propto 1/a$ with $V\propto a^3$. And I understand why entropy is conserved but not energy.}
\item{Marlene from Jena says: Due to the Hubble-expansion, every point is in recession motion with respect to every other point. If a photon gets scattered into your direction, the scattering particle will necessarily move away from you, leading to a lower perceived energy and a larger wavelength. It's important to view it like that because a photon gas can not change its state without interaction due to the linearity of electrodynamics, and this argument shows that it's a kinematic effect: It's a similarity transform of the Planck-spectrum.}
\item{Lorenzo from Florence says: It's important for the Planck-spectrum that the mean particle separation and the thermal wavelength are identical. You can only arrange for that if $T\propto 1/a$ if the particle separation increases $\propto a$. I'm only assuming that the number of photons is conserved, but not their energy, and that everything stays in equilibrium.}
\end{enumerate}

\item \question{CMB as a source of energy}\\
A Carnot-engine converts thermal energy taken from two reservoirs at different temperatures into mechanical energy at the efficiency $\eta = 1-T_2/T_1$.
\begin{enumerate}[(a)]
\item{Estimate if one can use the temperature anisotropies in the CMB of $\Delta T/T\simeq 10^{-5}$ to generate mechanical energy. How much power could you realistically generate? Construct a machine that converts radiation power into mechanical energy.}
\item{Could you use the time evolution of the CMB-temperature for this purpose? You know already that $T\propto 1/a$, so please construct a machine that produces energy from the CMB.}
\item{Why does a solar cell transform the radiation from the Sun into electrical energy? One might argue that the Planck-spectrum is that of thermal equilibrium in which case the mechanical work is zero: Due to the first law, mechanical work can not be performed in thermal equilibrium. (please be careful: trick question)}
\end{enumerate}

\end{enumerate}
\end{document}
