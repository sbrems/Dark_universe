\documentclass[a4paper,12pt]{article}

\usepackage{amssymb}
\usepackage{amsmath}
\usepackage{amsfonts}
\usepackage{txfonts}
%\usepackage{upgreek}
\usepackage{graphicx}
%\usepackage{siunitx}
\usepackage{enumerate}
\usepackage[left=2cm,right=2cm,top=2cm,bottom=2cm]{geometry}

%\newcommand{\question}[2]{\textbf{\textit{#1}}\quad{\footnotesize\textit{(#2 points)}}\\[3mm]}
\newcommand{\question}[1]{\textbf{\textit{#1}}}
\newcommand{\points}[1]{\quad{\footnotesize\textit{(#1 points)}}}
\newcommand{\point}{\quad{\footnotesize\textit{(1 point)}}}
\newcommand{\HRule}{\rule{\linewidth}{0.3mm}}
\newcommand{\dd}{\mathrm{d}}
%\renewcommand{\pi}{\uppi}
\newcommand{\ii}{\mathrm{i}}
\renewcommand{\thefootnote}{\normalsize\fnsymbol{footnote}}
\DeclareMathOperator{\e}{e}

\renewcommand{\theequation}{\Roman{equation}}

\begin{document}
\pagestyle{empty}

\begin{center}
\LARGE \textbf{Astronomy from 4 perspectives: the Dark Universe}
\HRule
\end{center}
\begin{flushright}
prepared by: Florence participants and BMS
\end{flushright}
\begin{center}
{\Large \textbf{High-School exercises: Supernova-cosmology and dark energy}}
\end{center}
\vspace{5mm}

\begin{enumerate}[\itshape \bfseries 1.]

\item \question{Classical potentials including a cosmological constant}\\
The field equation of classical gravity including a cosmological dark
energy density $\lambda$ is given by
\begin{align}
\Delta\Phi = 4\pi G\,(\rho + \lambda)
\end{align}
\begin{enumerate}
\item{Solve the field equation for $3$ dimensions outside a
    spherically symmetric and static matter distribution $\rho$. \\
    The expression for the Laplace-operator in spherical coordinates for $3$ dimensions is: 

\begin{equation}
\Delta\Phi = \frac{1}{r^{2}}\frac{\partial}{\partial r}\left(r^{2}\frac{\partial\Phi}{\partial r}\right)
\end{equation}    
    Also, please set as the total baryon mass $M$
\begin{equation}
M = 4\pi\int_0^r\dd r^\prime\, (r^\prime)^2\,\rho(r^\prime)
\end{equation}
}
\item{Show, that both source terms individually give rise to power-law solutions for $\Phi(r)$.}
\item{Is there a distance where the baryon part from the $\rho$-terms
    is equal to dark energy part the $\lambda$-term?}
\item{Assuming a typical galaxy is formed by $10^{11}$ stars like
    the Sun each with a mass of $10^{30}$ kg, and a dark energy density
    of $\lambda\,=\,10^{-27}$ kg / m$^3$, find at which distance from a galaxy the
    dark energy dominates. How does it compare with the typical size
    of a Galaxy ($\sim 10^4$ pc)?}
\end{enumerate}

\item \question{Light-propagation in FLRW-spacetimes}\\
Photons travel along null geodesics, $\dd s^2=0$, in any spacetime. 
\begin{enumerate}[(a)]
\item{Please show that by introducing {\em conformal time $\tau$} in a suitable definition, one recovers Minkowskian light propagation $c\tau = \pm\chi$ in comoving distance $\chi$ and conformal time $\tau$ for FLRW-space times,
\begin{equation}
\dd s^2 = c^2\dd t^2 - a^2(t)\,\dd\chi^2,
\end{equation}
which we have assumed to be spatially flat for simplicity.}
\item{What's the relationship between conformal time $\tau$ and cosmic time $t$? What would the watch of a cosmological observer display?}
\item{Please compute the conformal age of the Universe given a Hubble function $H(a)$,
\begin{equation}
H(a) = H_0 \, a^{-\,\gamma} \hspace*{6mm} \text{with: } \gamma > 0
\end{equation}
%which is filled up to the critical density with a fluid with a fixed equation of state $w$.
}
\item{In applying $\dd s^2=0$ to the FLRW-metric we have assumed a radial geodesic - is this a restriction?}
\item{Draw a diagram of a photon propagating from a distant galaxy to us in conformal coordinates for a cosmology of your choice, with markings on the light-cone corresponding to equidistant $\Delta a$.}
\end{enumerate}


%\item \question{light-propagation in perturbed metrics}\\
%The weakly perturbed ($\left|\Phi\right|\ll c^2$) Minkowskian metric is given by 
%\begin{equation}
%\dd s^2 = \left(1+2\frac{\Phi}{c^2}\right) c^2\dd t^2 - \left(1-2\frac{\Phi}{c^2}\right)\dd x_i\dd x^i
%\end{equation}
%with the Newtonian potential $\Phi$. Please compute the effective speed of propagation $c^\prime = \dd\left|x\right|/\dd t$ for a photon following a null geodesic $\dd s^2=0$. Please Taylor-expand the expression in the weak-field limit $\left|\Phi\right|\ll c^2$: Can you assign an effective index of refraction to a region of space with a nonzero potential?





%\item \question{physics close to the horizon}\\
%Why is it necessary to observe supernovae at the Hubble distance $c/H_0$ to see the dimming in accelerated cosmologies? Please start at considering the curvature scale of the Universe: A convenient quantisation of curvature might be the Ricci-scalar $R = 6H^2(1-q)$ for flat FLRW-models.
%\begin{enumerate}[(a)]
%\item{Can you define a distance scale $d$ or a time scale from $R$?}
%\item{What happens on scales $\ll d$, what on scales $\gg d$?}
%\end{enumerate}


\item \question{Measure cosmic acceleration}\\
The luminosity distance $d_\mathrm{lum}(z)$ in a spatially flat FLRW-universe is given by
\begin{equation}
d_\mathrm{lum}(z) = (1+z)\int_0^z\mathrm{d}z^\prime\:\frac{1}{H(z^\prime)}
\end{equation}
with the Hubble function $H(z)$. 
We can prove this relation between the scale factor $a$ and the cosmological red-shift $z$:

\begin{equation}
a=\frac{1}{1+z} 
\end{equation}

So we can write the Hubble function $H(z)$:

%Let's assume that the Universe is filled with a cosmological fluid up to the critical density with a fluid with equation of state $w$, such that the Hubble function is
\begin{equation}
H(z) = H_0 \, (1+z)^{\gamma}
\end{equation}
\begin{enumerate}
\item{Find the limit value of $\gamma$ for an accelerating (and non-accelerating) universe.   

Use the deceleration parameter given by 
\[q=-\frac{\ddot{a}\,a}{\dot{a}^2}\] from the Hubble-function \[H=\frac{\dot{a}}{a}\] to prove it.}
\item{Please show that in accelerated universes supernovae appear systematically dimmer, because $d_\mathrm{lum}$ is always larger than in a non-accelerating universe.}
%\item{Is it true that $d_\mathrm{lum}$ is systematically smaller in a decelerating universe?}
%\item{Would the expression for $d_\mathrm{lum}$ still be valid if the universe was contracting instead of expanding? What correction would you need to apply?}
\end{enumerate}

\end{enumerate}
\end{document}
