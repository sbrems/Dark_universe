\documentclass[a4paper,12pt]{article}

\usepackage{amssymb}
\usepackage{amsmath}
\usepackage{amsfonts}
\usepackage{txfonts}
\usepackage{upgreek}
\usepackage{graphicx}
%\usepackage{siunitx}
\usepackage{enumerate}
\usepackage[left=2cm,right=2cm,top=2cm,bottom=2cm]{geometry}

%\newcommand{\question}[2]{\textbf{\textit{#1}}\quad{\footnotesize\textit{(#2 points)}}\\[3mm]}
\newcommand{\question}[1]{\textbf{\textit{#1}}}
\newcommand{\points}[1]{\quad{\footnotesize\textit{(#1 points)}}}
\newcommand{\point}{\quad{\footnotesize\textit{(1 point)}}}
\newcommand{\HRule}{\rule{\linewidth}{0.3mm}}
\newcommand{\dd}{\mathrm{d}}
\renewcommand{\pi}{\uppi}
\newcommand{\ii}{\mathrm{i}}
\renewcommand{\thefootnote}{\normalsize\fnsymbol{footnote}}
\DeclareMathOperator{\e}{e}

\renewcommand{\theequation}{\Roman{equation}}

\begin{document}
	\pagestyle{empty}
	
	\begin{center}
		\LARGE \textbf{Astronomy from 4 perspectives: the Dark Universe}
		\HRule
	\end{center}
	\begin{flushright}
		prepared by: Florence participants and BMS
	\end{flushright}
	\begin{center}
		{\Large \textbf{High-School exercises: Supernova-cosmology and dark energy}}\\
		\vspace*{2mm}
		{\Large \textbf{Solutions}}
		
	\end{center}
	\vspace{5mm}
	
	\begin{enumerate}[\itshape \bfseries 1.]
	
			\item \question{Classical potentials including a cosmological constant}\\
		The field equation of classical gravity including a
                cosmological dark energy density $\lambda$ is given by
		\begin{equation}
		\Delta\Phi = 4\pi G(\rho + \lambda)
		\end{equation}
		
		\begin{enumerate}
		\item Field calculation\\
	
		Now it is possible to simply integrate the field equation starting with:
		\begin{align}
		  \Delta\Phi&=\frac{1}{r^{2}}\frac{\partial}{\partial r}\left(r^{2}\frac{\partial\Phi}{\partial r}\right)\\
		  &=4\pi G(\rho(r)+\lambda)\\
		  r^{2}\frac{\partial\Phi}{\partial
                    r}&=\int_0^r\textrm{d}r'\left(4\pi G[\left(r'\right)^{2}\rho\left(r'\right)+\left(r'\right)^{2}\lambda]\right)\\
		  &=G \, M \, + \,  \left(\frac{4\pi\, G \, r^{3}}{3}\right)\,\lambda\\
		  \frac{\partial\Phi}{\partial r}
		  &=G \left(\frac{M}{r^{2}}+\frac{4 \pi \, r}{3} \, \lambda\right)\\
		  \Phi&=G\left(-\frac{M}{r}+\frac{4 \pi \, \lambda\, r^2}{6}\right)
		\end{align}
		
%		\item power-law solutions\\
%		Following the calculation one may see that each source term corresponds to an individual power-law:
%		\begin{align*}
%		  C(n)G\rho(r) ~~~ &\Rightarrow ~~~ -\frac{GM}{r}\\
%		  \lambda ~~~ &\Rightarrow ~~~ G\frac{\lambda r^2}{6}
%		\end{align*}
		
		\item "Equilibrium"\\
		
		To find an equilibrium distance one must set \(\Phi\left(r_\textrm{eq}\right)=0\)
		
		\begin{align}
		  \frac{M}{r_\textrm{eq}}&= \, \frac{4 \pi \, \lambda \, r_\textrm{eq}^2}{6}\\
		  \frac{4 \pi \, \lambda \, r_\textrm{eq}^3}{6}&=M
		\end{align}
		from which follows immediatly:
		\begin{equation}
		  r_\textrm{eq}=\sqrt[3]{\frac{6}{4 \pi} \, \frac{M}{\lambda}}
		\end{equation}
		 
		If one inputs the number one gets $ r_\textrm{eq} $ few  Mpc one hundred times larger than the size of a galaxy.
		We can really observe the Dark-Energy effect at this distance? No, because this value of $r_{eq}$ is the typical value of intergalactic distances.
        
        \end{enumerate}
		
		\item \question{Light-propagation in FLRW-spacetimes}\\
		Photons travel along null geodesics, $\dd s^2=0$, in any spacetime. 
		\begin{enumerate}
			\item Let us do the following substitution
			 \[dt \rightarrow a(t)d\tau\] 
			 then the line element can be written
                          
                          
                          \[ds^2=a(t)^2\, \left[ c^2d\tau^2-d\chi^2\right] \] 
                          and the equation of the null-geodesic will be
                          
                          \[d\chi=\pm cd\tau\].
			\item The cosmic time is the time measured by
                          a cosmic observer synchronized for $t=0$ 
 \begin{align}
   t=\int_0^t dt' = \int_0^a\frac{da'}{\dot{a}'}
   \end{align}
The conformal time is tied to the time interval over which an observer
at $t=t_0$ sees to happen an event in the past at time $t$. Now at
$t=t_0$ this will coincide with the cosmic time, ence it will be
affected by cosmic time dilation.
                          \begin{align}
                            \tau(t)=\int_0^t \frac{dt'}{a(t')} = \frac{1}{a(t)}
                              \int_0^t \frac{a(t)}{a(t')} dt' > \frac{t}{a(t)}
                          \end{align}
			\item Now for the given metric:
                          \begin{align}
                            H=\frac{\dot{a}}{a}= H_0 \, a^{-\,\gamma}
                            \Rightarrow \dot{a} = H_0 \, a^{1-\gamma}
                            \end{align}
				We can solve this equation:

			\begin{align}
			\dfrac{da}{dt} \, = \, H_0 \, a^{1 - \gamma} \\
			a^{\gamma - 1} \, da = H_0 \, dt \\
			\int_0^a (a')^{\gamma - 1} da' = H_0 \, \int_0^t dt = H_0 \, t \\
			\frac{a^{\gamma}}{\gamma} \, = \, H_0 \, t \\
			a(t) \, = \, \sqrt[\gamma]{\gamma \, H_0 \, t}
			\end{align}

So we can obtain for the Age of the Universe			
			
							\begin{align}
							\tau_H=\int_0^t \frac{dt'}{a(t')} =
                            \int_0^1 \frac{da}{\dot{a}\,a} =
                            \frac{1}{H_0}\int_0^1 a^{\gamma \, - \, 2}\,da =  									\frac{1}{H_0}\frac{1}{\gamma - 1}
                            \end{align}

			\item Isotropy of the universe ensures us that it is not.
		\end{enumerate}
		
		
%		\item \question{light-propagation in perturbed metrics}\\
%		\begin{align}
%		ds^2=\left(1+2\frac{\Phi}{c^2}\right)c^2dt^2-\left(1-2\frac{\Phi}{c^2}\right)dx^2 
%		\end{align}
%		With $ds^2=0$:
%		\begin{align}
%		\left( 1+\frac{2\Phi}{c^2}\right)c^2dt^2 &= \left(1-\frac{2\Phi}{c^2}\right) dx^2\\
%		\frac{dx}{dt}&=\pm c\sqrt{\frac{1+\frac{2\Phi}{c^2}}{1-\frac{2\Phi}{c^2}}}
%		\end{align}
%		With $\frac{1}{1-\epsilon}\approx 1+\epsilon$ for small $\epsilon$:
%		\begin{align}
%		\frac{dx}{dt}\approx\pm c\left(1+\frac{2\Phi}{c^2}\right)
%		\end{align} 
%		For a non-zero $\Phi$ this is not equal to $c$! \\
%		We assign an effective index of refraction by:
%		\begin{align}
%		n(\Phi)=\frac{dx/dt}{c}\approx \left(1+\frac{2\Phi}{c^2}\right)
%		\end{align}
		

%	\item \question{physics close to the horizon}\\
%		Why is it necessary to observe supernovae at the Hubble distance $c/H_0$ to see the dimming in accelerated cosmologies? Please start at considering the curvature scale of the Universe: A convenient quantisation of curvature might be the Ricci-scalar $R = 6H^2(1-q)$ for flat FLRW-models.
%		\begin{enumerate}
%			\item The Dimension of the Ricci-scalar is $1/s^2$ thus we can define a time and a distance scale by:
%			$$ \tau = 1/\sqrt{R} \ \ \ \mathrm{and} \ \ d = c/\sqrt{R} \approx c/H_0$$
%			which gives the curvature scale of the Universe. 
%			\item To observe supernovae dimming caused by accelerated cosmic expansion the supernova distance had to be about (or larger than) the curvature scale, because at distances $<<d$ the different cosmological distance measures converge. \\
%			For illustration see: \textit{https://en.wikipedia.org/wiki/Distance\_measures\_(cosmology)}
%		\end{enumerate}
		
		
		\item \question{Measure cosmic acceleration}\\
		The luminosity distance $d_\mathrm{lum}(z)$ in a spatially flat FLRW-universe is given by
		\begin{equation}
		d_\mathrm{lum}(z) = (1+z)\int_0^z\mathrm{d}z^\prime\:\frac{1}{H(z^\prime)}
		\end{equation}
		with the Hubble function $H(z)$:
		\begin{enumerate}
			\item
					By definition of the Hubble Function $H$ and deceleration parameter $q$:
					\begin{align*}
						H=\frac{\dot{a}}{a} \\
						q=-\frac{\ddot{a}a}{\dot{a}^2}
					\end{align*}
		It follows
		\begin{align*}
		\dot{H}&=\frac{\ddot{a}a-\dot{a}^2}{a^2}=\frac{\ddot{a}a}{a^2}-H^2		
		\end{align*}
		So we get:
		\begin{align*}
		\frac{\dot{H}}{H^2}&=\frac{\ddot{a}a}{\dot{a}^2}-1=-\,q -\, 1\\
		q&=-\left(\frac{\dot{H}}{H^2}+1\right)	
		\end{align*}
	
		If we use $H(a) \, = \, H_0 \, a^{- \, \gamma}$:
		\begin{align*}
		\dot{H} &= - \, \gamma \, H_0\cdot a^{- \, \gamma - 1}\cdot \dot{a}\\
			  &= - \,\gamma\, H_0 \cdot a^{- \, \gamma}\cdot\frac{\dot{a}}{a}\\
			  &= - \gamma \, H^2
		\end{align*}	
		And then we get:

		\begin{align*}
		\frac{\dot{H}}{H^2} \, = \, - \, \gamma
		\end{align*}				
		Now we can substitute in the deceleration parameter expression:
		\begin{align*}
		q=-\left(- \, \gamma +1\right)= \gamma \, - 1
		\end{align*}
		and obviously
		\begin{align*}
		q<0 \,\,\text{ for } \gamma < 1\\
		q>0 \,\,\text{ for } \gamma > 1
		\end{align*}	
										
			\item
				First, we consider the case $\gamma = 1$:
				\begin{align*}
			H=H_0(1+z)^\gamma = H_0(1+z)\\
\end{align*}

Then we find this expression for the luminosity distance	
\begin{align*}
d_{lum,1}&=(1+z)\int_0^z \! \frac{1}{H(z')} \, \mathrm{d}z'\\
			&=(1+z)\int_0^z \! \frac{1}{H_0(1+z')} \, \mathrm{d}z'\\
			&=\frac{1+z}{H_0}\,ln(1+z)
\end{align*}						
	Now, we consider the case $\gamma < 1$ (accelerating universe):
	\begin{align*}
				d_{lum,2}&=(1+z)\int_0^z \! \frac{1}{H(z')} \, \mathrm{d}z'\\
				&=\frac{1+z}{H_0}\int_0^z \! (1+z')^{- \gamma} \, \mathrm{d}z'\\
				&=\frac{1+z}{H_0}\left[(1+z')^{- \gamma + 1}\cdot \frac{1}{- \gamma + 1}\right]_0^z\\
				&=\frac{1+z}{H_0}\left(\frac{1}{- \gamma + 1}\right)\left[(1+z)^{- \gamma + 1}-1\right]
\end{align*}	
It follows:
$d_{lum_2}(z)>d_{lum_1}(z)$, because the exponent $(- \gamma + 1)$ is positive $(\gamma < 1)$,
so $d_{lum_2}(z)$ is growing faster, than the logarithmic function $d_{lum_1}(z)$.		
%			\item Yes, as long as the universe is flat. Just try to plot the two functions.
%			\item The formula still apply,. But in a
%                          contracting universe one would see light
%                          that is blue-shifted and not red-shifted. Su
%                          $z<0$. But Obviiously $z>-1$ otherwhise one
%                          would get negative wavelengths (frequancies)
%                          that make no sense. As a consequence, the
%                          origin of time in a contracting universe
%                          cosrresponds to $z=-1$ for a universe
%                          contracting from infinity. Hence the limit
%                          of integration must be corrected accordingly.
		\end{enumerate}
		
	\end{enumerate}
		
\end{document}
